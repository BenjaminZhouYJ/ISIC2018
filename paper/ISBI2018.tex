% Template for ISBI-2018 paper; to be used with:
%          spconf.sty  - ICASSP/ICIP LaTeX style file, and
%          IEEEbib.bst - IEEE bibliography style file.
% --------------------------------------------------------------------------
\documentclass{article}
\usepackage{spconf,amsmath,graphicx}

% Example definitions.
% --------------------
\def\x{{\mathbf x}}
\def\L{{\cal L}}

% Title.
% ------
\title{Lesion Attributes Segmentation for Melanoma Detection with Deep Learning}
%
% Single address.
% ---------------
%\name{Author(s) Name(s)\thanks{Thanks to XYZ agency for funding.}}
%\address{Author Affiliation(s)}
%
% For example:
% ------------
%\address{School\\
%	Department\\
%	Address}
%
% Two addresses (uncomment and modify for two-address case).
% ----------------------------------------------------------
%\twoauthors
%  {A. Author-one, B. Author-two\sthanks{Thanks to XYZ agency for funding.}}
%	{School A-B\\
%	Department A-B\\
%	Address A-B}
%  {C. Author-three, D. Author-four\sthanks{The fourth author performed the work
%	while at ...}}
%	{School C-D\\
%	Department C-D\\
%	Address C-D}
%
% More than two addresses
% -----------------------
 \name{
 Eric Z. Chen$^{1}\sthanks{To whom correspondence should be addressed}$ 
 \qquad Xu Dong$^{2}$
 \qquad Junyan Wu$^{3}$ 
 \qquad Hongda Jiang$^{4}$ 
 \qquad Xiaoxiao Li$^{5}$
 \qquad Ruichen Rong$^{6}$
 }

\address{
$^{1}$ Dana-Farber Cancer Institute, Boston, MA, USA \\
$^{2}$Virginia Tech, Blacksburg, VA, USA \\
$^{3}$Cleerly Inc, New York City, New York, USA \\
$^{4}$East China University of Science and Technology, Shang Hai, China \\
$^{5}$Yale University, New Haven, CT, USA\\
$^{6}$UT Southwestern Medical Center, Dallas, TX, USA
}



\begin{document}
%\ninept
%
\maketitle
%
\begin{abstract}
Melanoma is the most deadly form of skin cancer worldwide. Many efforts have been made for early detection of melanoma. The International Skin Imaging Collaboration (ISIC) hosted the 2018 Challenges to help the diagnosis of melanoma based on dermoscopic images. In this paper, we describe our solutions for the task 2 of ISIC 2018 Challenges. We present two deep learning approaches to automatically detect lesion attributes of melanoma, one is a multi-task U-Net model and the other is a Mask R-CNN based model. Our multi-task U-Net model achieved a Jaccard index of 0.433 on official test data, which ranks the 5th place on the final leaderboard. The code for our solutions is publicly available. 
\end{abstract}
%
\begin{keywords}
Skin cancer, deep learning
\end{keywords}
%
\section{Introduction}
\label{sec:intro}

%% skin cancer
Skin cancer is one of the most common cancers worldwide and more than one million skin cancers have been diagnosed in the United States each year. Melanoma is the most dangerous form of skin cancer, which causes over 9,000 deaths each year \cite{ucsw2013united}. Melanoma in the late stage can often spread to other parts of the body and thus it is difficult to treat and usually can be fatal. However, melanoma in the early stage is treatable and the majority can be cured. Many efforts have been made to detect melanoma in the early stage. Dermoscopy is one noninvasive method commonly used in the healthcare to examine pigmented skin lesion. It can generate high-resolution images of the lesion regions on the skin. To diagnose melanoma, it still requires the dermatologist to evaluate the images based on several skin lesion patterns. The common dermoscopic attributes are pigment network, negative network, streaks, milia-like cysts, and globules \citep{mishra2016overview}. Automatic detection of those skin lesion attributes can be a tremendous help for early melanoma diagnosis. 

Towards this goal, the International Skin Imaging Collaboration (ISIC) hosted the 2018 Challenges to diagnose melanoma automatically based on dermoscopic images. The 2018 challenges include three tasks. The first task is to predict lesion segmentation boundaries in dermoscopic images. The second task is to predict the locations of five skin lesion patterns (i.e., dermoscopic attributes) in dermoscopic images, which are pigment network, negative network, streaks, milia-like cysts, and globules. The third task is to predict seven disease categories in dermoscopic images, which includes melanoma, melanocytic nevus, basal cell carcinoma, actinic keratosis / Bowen’s disease, benign keratosis, dermatofibroma and vascular lesion. The three tasks mimic the steps for lesion analysis performance by dermatologists in the clinic settings.  From the previous years' competitions, it has been reported that far less participation in the second task of lesion pattern prediction than in other tasks \citep{codella2018skin}. It seems that the second task is the most difficult one and this motivated us to focus on this problem instead of the other two. 

%% deep learning, segmentation
Deep learning, especially convolutional neural network (CNN), has been widely applied to solve many problems in computer vision. Various CNN based models have been developed for object classification and detection, such as VGG \citep{simonyan2014very}, ResNet \citep{he2016deep}, Inception \citep{szegedy2017inception}, DenseNet \citep{huang2017densely}, and more \citep{liu2017survey}. For image segmentation, FCN \citep{long2015fully}, U-Net \citep{ronneberger2015u}, Mask R-CNN \citep{he2017mask} are some of the most used deep learning models.  Many deep learning based approaches have been applied for melanoma detection \citep{mishra2016overview}. However, most of those researches applied deep learning models to either classify skin diseases \citep{haenssle2018man, yu2018acral} or segment whole lesion regions \citep{zhang2017melanoma}, which are corresponding to the task 1 and task 3 in ISIC 2018 Challenges. To our knowledge, currently there is no deep learning based approach has been applied to segment specific lesion attributes as in the task 2 of ISIC 2018 Challenges. 

%% our methods
In this paper, we describe our solutions for the task 2 of ISIC 2018 Challenges. We present two deep learning approaches to automatically detect lesion attributes of melanoma. One is a U-Net based approach and the other is a Mask R-CNN based approach. In the U-Net model, we replaced the encoder part of the U-Net with a pretrained VGG16 network \citep{shvets2018automatic}. In the middle layer and the last layer of the U-Net, we added two classification heads to classify the empty masks versus the non-empty masks.  Our multi-task U-Net model achieved a Jaccard index of 0.433 on official test data, which ranks the 5th place on the final leaderboard. The code for our solutions is publicly available at https://github.com/chvlyl/ISIC2018. 

\section{Methods}
\label{sec:Methods}

All printed material, including text, illustrations, and charts, must be kept
within a print area of 7 inches (178 mm) wide by 9 inches (229 mm) high. Do
not write or print anything outside the print area. The top margin must be 1
inch (25 mm), except for the title page, and the left margin must be 0.75 inch
(19 mm).  All {\it text} must be in a two-column format. Columns are to be 3.39
inches (86 mm) wide, with a 0.24 inch (6 mm) space between them. Text must be
fully justified.

\section{PAGE TITLE SECTION}
\label{sec:pagestyle}

The paper title (on the first page) should begin 1.38 inches (35 mm) from the
top edge of the page, centered, completely capitalized, and in Times 14-point,
boldface type.  The authors' name(s) and affiliation(s) appear below the title
in capital and lower case letters.  Papers with multiple authors and
affiliations may require two or more lines for this information.

\section{TYPE-STYLE AND FONTS}
\label{sec:typestyle}

To achieve the best rendering both in the proceedings and from the CD-ROM, we
strongly encourage you to use Times-Roman font.  In addition, this will give
the proceedings a more uniform look.  Use a font that is no smaller than nine
point type throughout the paper, including figure captions.

In nine point type font, capital letters are 2 mm high.  If you use the
smallest point size, there should be no more than 3.2 lines/cm (8 lines/inch)
vertically.  This is a minimum spacing; 2.75 lines/cm (7 lines/inch) will make
the paper much more readable.  Larger type sizes require correspondingly larger
vertical spacing.  Please do not double-space your paper.  True-Type 1 fonts
are preferred.

The first paragraph in each section should not be indented, but all the
following paragraphs within the section should be indented as these paragraphs
demonstrate.

\section{MAJOR HEADINGS}
\label{sec:majhead}

Major headings, for example, "1. Introduction", should appear in all capital
letters, bold face if possible, centered in the column, with one blank line
before, and one blank line after. Use a period (".") after the heading number,
not a colon.

\subsection{Subheadings}
\label{ssec:subhead}

Subheadings should appear in lower case (initial word capitalized) in
boldface.  They should start at the left margin on a separate line.
 
\subsubsection{Sub-subheadings}
\label{sssec:subsubhead}

Sub-subheadings, as in this paragraph, are discouraged. However, if you
must use them, they should appear in lower case (initial word
capitalized) and start at the left margin on a separate line, with paragraph
text beginning on the following line.  They should be in italics.

\section{PRINTING YOUR PAPER}
\label{sec:print}

Print your properly formatted text on high-quality, 8.5 x 11-inch white printer
paper. A4 paper is also acceptable, but please leave the extra 0.5 inch (12 mm)
empty at the BOTTOM of the page and follow the top and left margins as
specified.  If the last page of your paper is only partially filled, arrange
the columns so that they are evenly balanced if possible, rather than having
one long column.

In \LaTeX, to start a new column (but not a new page) and help balance the
last-page column lengths, you can use the command ``$\backslash$pagebreak'' as
demonstrated on this page (see the \LaTeX\ source below).

\section{PAGE NUMBERING}
\label{sec:page}

Please do {\bf not} paginate your paper.  Page numbers, session numbers, and
conference identification will be inserted when the paper is included in the
proceedings.

\section{ILLUSTRATIONS, GRAPHS, AND PHOTOGRAPHS}
\label{sec:illust}

Illustrations must appear within the designated margins.  They may span the two
columns.  If possible, position illustrations at the top of columns, rather
than in the middle or at the bottom.  Caption and number every illustration.
All halftone illustrations must be clear black and white prints.  If you use
color, make sure that the color figures are clear when printed on a black-only
printer.

Since there are many ways, often incompatible, of including images (e.g., with
experimental results) in a \LaTeX\ document, below is an example of how to do
this \cite{Lamp86}.

% Below is an example of how to insert images. Delete the ``\vspace'' line,
% uncomment the preceding line ``\centerline...'' and replace ``imageX.ps''
% with a suitable PostScript file name.
% -------------------------------------------------------------------------
%\begin{figure}[htb]
%
%\begin{minipage}[b]{1.0\linewidth}
%  \centering
%  \centerline{\includegraphics[width=8.5cm]{image1}}
%%  \vspace{2.0cm}
%  \centerline{(a) Result 1}\medskip
%\end{minipage}
%%
%\begin{minipage}[b]{.48\linewidth}
%  \centering
%  \centerline{\includegraphics[width=4.0cm]{image2}}
%%  \vspace{1.5cm}
%  \centerline{(b) Results 3}\medskip
%\end{minipage}
%\hfill
%\begin{minipage}[b]{0.48\linewidth}
%  \centering
%  \centerline{\includegraphics[width=4.0cm]{image3}}
%%  \vspace{1.5cm}
%  \centerline{(c) Result 4}\medskip
%\end{minipage}
%%
%\caption{Example of placing a figure with experimental results.}
%\label{fig:res}
%%
%\end{figure}



% To start a new column (but not a new page) and help balance the last-page
% column length use \vfill\pagebreak.
% -------------------------------------------------------------------------
\vfill
\pagebreak


\section{FOOTNOTES}
\label{sec:foot}

Use footnotes sparingly (or not at all!) and place them at the bottom of the
column on the page on which they are referenced. Use Times 9-point type,
single-spaced. To help your readers, avoid using footnotes altogether and
include necessary peripheral observations in the text (within parentheses, if
you prefer, as in this sentence).


\section{COPYRIGHT FORMS}
\label{sec:copyright}

You must include your fully completed, signed IEEE copyright release form when
you submit your paper. We {\bf must} have this form before your paper can be
published in the proceedings.  The copyright form is available as a Word file,
a PDF file, and an HTML file. You can also use the form sent with your author
kit.

\section{REFERENCES}
\label{sec:ref}

List and number all bibliographical references at the end of the paper.  The references can be numbered in alphabetic order or in order of appearance in the document.  When referring to them in the text, type the corresponding reference number in square brackets as shown at the end of this sentence \cite{C2}.

% References should be produced using the bibtex program from suitable
% BiBTeX files (here: strings, refs, manuals). The IEEEbib.bst bibliography
% style file from IEEE produces unsorted bibliography list.
% -------------------------------------------------------------------------
\bibliographystyle{IEEEbib}
\bibliography{ref}

\end{document}
